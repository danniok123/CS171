\documentclass[11pt]{article}
\newcommand\floor[1]{\lfloor#1\rfloor}
\usepackage{amsmath}

%\usepackage{customizations}
\usepackage{listings}


\textwidth=15cm
\oddsidemargin=0.7cm
\voffset=-1cm
\textheight=20cm
\usepackage{xcolor}


\begin{document}


\title{
  CS 171: Visualization \\
  \Large{Homework 3 (Due September 25th, 2017 at 11:59pm)}
}
\date{}

\maketitle

\vspace{-1cm}


\section*{Sketch Analysis}
I chose sketch 2.
\begin{enumerate}
  \item \textbf{For each data type, describe how it is encoded in your redesign using Bertin's marks and channels.} \newline
  	In my redesign of the graph, I wanted to emphasize size and also color contrast. By using just one color and varying the saturation levels, it wouldn't make the visualization as confusing to understand. I also utilized Bertin's shape and wanted to combine a pie chart in the visualization.
	
  \item \textbf{How are the perceptual channels contrast and color used in your chosen sketch?} \newline
  As mentioned before, contrast and color are used in my sketch by just using one color and varying the contrast. 
  
  \item \textbf{How are Tufte's design principles used or violated in your sketch?} \newline
  I believe I adhere to Tufte's design principles pretty well because I used color where color was necessary, I tried to guesstimate as good as possible to figure out the percentages of causes of death for each category, and there isn't unnecessary junk in my graph. I probably could have chosen a better color however I have so few pens. 
  
\end{enumerate}


\end{document}
