\documentclass[11pt]{article}
\newcommand\floor[1]{\lfloor#1\rfloor}
\usepackage{amsmath}

%\usepackage{customizations}
\usepackage{listings}


\textwidth=15cm
\oddsidemargin=0.7cm
\voffset=-1cm
\textheight=20cm
\usepackage{xcolor}

\renewcommand{\labelenumii}{\Roman{enumii}}

\begin{document}


\title{
  CS 171: Visualization \\
  \Large{Homework 2 (Due September 18th, 2017 at 11:59pm)}
}
\date{}

\maketitle

\vspace{-1cm}


\section*{Design Analysis / Critique}
I chose to look at Visualization B.

\subsection{Audience}
I believe the audience intended for this visualization is for people that are interested in trying to get a Nobel Prize and understanding the different characteristics of people who did received them.

\subsection{Visualization Questions}
This graph tries to give a visualization of the different qualifications of people who got Nobel Prizes from 1901 to 2012. It's broken down into many categories such as age, concentration, university attended, gender, hometowns, and education level. By looking at the graphical representation, the author hopes people can get a general sense of the data and be able to analyze it.

\subsection{Design Principles}

\begin{enumerate}
  \item Good
  \begin{enumerate}
    \item Tells you specific qualifications such as age and education level
    \item Disregarding whether it's understandable or not, it is visually appealing
    \item Gives you average age for each category
  \end{enumerate}
  \item Bad
  \begin{enumerate}
    \item Unable to understand at first glance
    \item Too many distracting visuals
    \item Slanted visualization is unnecessary
  \end{enumerate}
\end{enumerate}

\subsection{Dislikes/Likes}
I don't like the visualization because I feel that for graphs, I should be able to understand it with a first glance. I also don't like how certain visualizations are placed. As noted above, I don't think it's necessary to slant the visual because now I have to cock my head which is unnecessary. What I do like is that there are many categories, because it really specifies certain attributes of Nobel Prize winners.

\subsection{Data-Ink Ration and Other Suggestions}
I think the data-ink ration can be improved by separating the visual into multiple graphs, so as to not mix so much information in one graph. By doing this, the author will be able to present the data more clearly and not confuse the readers. By doing this, this will be able to decrease the data-ink.


\end{document}
